First, MoSCoW requirements are divided into two sections.
\begin{itemize}
  \item Functional requirements - Describe what a product must do.
  \item Non-functional requirements - Describe how the system works.
\end{itemize}
Furthermore, since we have a limited amount of time, we decided to prioritize the functional requirements in order to make progress
and keep to deadlines. MoSCoW is a prioritization technique for helping to understand and manage priorities. The letters stand
for
\begin{itemize}
  \item \textbf{M}ust have - Functional requirements that are critical to the current delivery time-box for a
                              successful project
  \item \textbf{S}hould have - Functional requirements that are important but not as time-critical as the must-
                                have requirements. Therefore, they can be delivered in the future delivery time-box.
  \item \textbf{C}ould have - Functional requirements that are desirable but not necessary. Usually, these
                                requirements improve user or customer experience.
  \item \textbf{W}on’t have - Requirements that are perceived as the least-critical by stakeholders. This
                                category includes requirements that are dropped and considered for inclusion in a future
                                version of the project.
\end{itemize}
