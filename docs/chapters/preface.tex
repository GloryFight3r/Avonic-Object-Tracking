This project is done as part of the CSE2000 ``Software Project'' course of the Computer Science and Engineering Bachelor program
at Delft University of Technology.
Avonic\cite{avonic} is a company based in Delft with over fifteen years of experience in the field of audiovisual solutions.
They specialize in producing pan-tilt-zoom (PTZ) cameras for markets like education and enterprise all around the world.


As there are not many working, commercially available solutions that integrate a video-capturing device with a microphone array,
Avonic has decided to investigate the possibility of combining a camera and a \gls{microphone} to increase the accuracy of tracking speakers.


Our team found the idea and motivation of the project intriguing, and as computer science students with diverse backgrounds,
we decided to collaborate with Avonic.
Together we believe that automating the process of speaker tracking based on audio information will increase the quality of lecture and conference recordings,
and eliminate potential human errors in them.


We would like to thank our supervisor A. Kahawati for his advice and representation of Avonic’s views on the matters of the project,
our Teaching Assistant S. Van der Voort for the invaluable feedback given to us each week,
our coach Q. Song for the feedback that helped us extend the project and guided us to revise our design decisions,
and our Technical Writing coach P. Van den Hoven for guidance on the report.
\vspace{1.5cm}
\begin{flushleft}
    Delft, 25 June 2023


    Izzy van der Giessen, Petr Khartskhaev, Yehor Kozyr,\\
    Borislav Semerdzhiev, Ivan Smilenov \\
\end{flushleft}
