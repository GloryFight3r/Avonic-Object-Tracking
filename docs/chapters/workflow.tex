\section{Values and teamwork approach}
Our team has very specific values which we always aim to follow no matter the circumstances. This is how we ensure mutual trust and good understanding in the team.

The most important values we appreciate are hard work and authority. It goes without saying that we value hard work since we all aim to complete our project in the best possible way.
Authority is important to us because we should always respect Avonic and the university staff since they are the people who evaluate our work.
Another crucial value in our team is humor. It is always better when working becomes a pleasure and not a boresome duty.
\section{Workflow practices}
\subsection{Approaches}
To ensure a good workflow we decided to use Agile and more specifically -- Scrum.
Scrum gives teams the ability to cope with dynamic priorities and requirements by introducing sprint planning, retrospectives, and reviews.
This way of planning helps to organize collaborative work faster and more precisely.

As we had regular sprint planning session, planning poker practice fitted well with them.
However, this did not prove to work well immediately, as our overall vision of the issues seemed to be not accurate in the beginning.
Our improved became better in later stages of the project.

We also decided to employ TDD (test-driven development) as it has been shown to reduce number of bugs in the code.
It is also useful to write simple yet reliable code, and motivates a developer to be more conscious of the design decisions and edge cases.

\subsection{Requirements for new functionality}
In order to accept a new functionality we came up with three requirements:
\begin{enumerate}
	\item Has to have a test coverage of at least 80\%
	\item Has to have a clear description of the functionality
	\item Must not break any other functionality
\end{enumerate}

\subsection{Definition of done}
Our project is considered done when all the listed must-have requirements are fulfilled. Any other requirements we manage to solve are a bonus. Realistically we will not be able to implement all of them but we aim to complete as many as possible.
We also have a definition of done for all of our Gitlab issues. It describes the behavior that should be possible upon completion of this issue. Issues should be closed only after this definition of done has been satisfied.
\subsection{Additional tools}
\begin{itemize}
	\item \verb|pylint| for styling purposes
	\item \verb|pytest| for basic testing, \verb|pytest-cov| to calculate coverage
	\item \verb|mutmut| for mutation testing
	\item \verb|git| for version control using GitLab as the server
	\item CI (continuous integration) for quality assurance - runs on GitLab after each change
\end{itemize}

\subsection{Communication}
We make sure all work is done on time and in an upstanding way by meeting every day. This happens either on campus, in Avonic's office, or on Discord as a backup. We discuss our progress and help each other. If any problem arises we brainstorm and come up with solutions together.
Mattermost serves as a communication channel with our TA.
WhatsApp is our backup if everything else fails.

\section{Collaboration with Avonic}
Avonic was open to collaboration and in this section, we give some insights into how it went:

Every week we met with Mr. Kahawati to discuss the progress we are had since our last meeting.
During those meetings, we would discuss feedback on implemented features and their possible extensions.

As discussed in \hyperref[sec:possible-pitfalls]{possible pitfalls}. we could work in the office only a limited amount of time. Avonic gave us a camera and set up VPN for us, so we could continue work on project even when were not in the office.

In the beginning of the project, we experienced some problems with time outs of the camera connection over the network, but people in Avonic helped us figure out that it was caused by network settings and not us.

\section{Testing}

\section{Challenges}
In this section, we discuss the challenges we faced as a group and how we solved them.
\subsection{WebUI design with multiple tabs}
From the beginning of the project, the user interface was growing on a single page.
If a new feature was developed, it also added a new section or a button to the web page.
As the size and the scale of the application grew, a new vision for the UI’s pages developed among a part of the team.

The changes have deleted a single-page design and introduced a multiple-tab design - with only a certain set of sections would display once at a time.
You would need to switch to a different tab via the menu at the opt of the page, to go to a different section.

The decision was not discussed within a team prior to implementation.
When a team member presented their changes in the UI design before merging, opinions were divided about the UI, as it was not discussed prior to implementation.
Part of the team liked it, while others did not.
Justification of both sides was centered around the comfort of using the web page - one part thought the previous solution was cluttered, while the other considered the new version to require more effort to use, as you would need to click through multiple tabs to execute a sequence of commands.

All of the steps from the Code of conduct of our team were taken to reach a consensus on this issue, but the group was still divided after 2 rounds of discussions.
As we could not reach the decision ourselves, we have decided to ask Mr. Kahawati, as he has a greater insight into what kind of UI customers would prefer.
As described in that meeting, the UI is most probably going to be used by technicians, so having everything on the same page is more practical for them, but the system can also be used by people with very little knowledge about cameras and such a big menu can overwhelm them.

Therefore, the decision was to combine both designs, by adding tabs in addition to a single-page design. By default, a user is presented with a page of the whole system’s functionality, but if they want to isolate certain aspects of it, they can click on a respective tab at the top of the page.
\textbf{THERE WILL BE AN IMAGE OF THE FINAL UI VERSION}

\subsection{Breaking production server}\label{subsection:breaking-production-server}
As discussed in subsection about the \hyperref[subsection:solution_webui]{WebUI}, we decided to use uWSGI for production mode.
This decision combined with poor quality assurance later back-fired in the situation below.
When more features that relied on threads were added, some of them started using mutable variables, in a way that change of it in one of threads, would affect another one.
But, as we found out, it does not work when the server is run with uWSGI, as it replaces multithreading with multiprocessing.
This leads to a lot of problems, as variables are not shared between processes.

We had not noticed it after a few merge requests that were rapidly merged because of the deadline.
We could not conduct a good quality assurance process of testing, which led to the fact that multiple features, like footage and information updates, that relied on mutability of variables stopped working .
As we did not have enough knowledge back in time, it took us a whole day of debugging to identify the problem.

After the issue was found, it was fixed and did not cause any problems later.
Since then, the quality assurance procedure was strengthened, to make sure that features work both in production and development modes.
