Below you find documentation of all of the endpoints provided
by our application.

\section{Page rendering}

\paragraph{GET /} \mbox{} \\
Render the main page of the WebUI.
This page includes everything our application has to offer in one neatly organised package

\paragraph{GET /camera-control} \mbox{} \\
Render a page that includes only the camera controls.

\paragraph{GET /microphone-control} \mbox{} \\
Render a page that includes only the microphone information.
That means the buttons to get its direction and whether someone is currently speaking.

\paragraph{GET /presets-and-calibration} \mbox{} \\
Render a page that includes only set-up for presets and calibration.

\paragraph{GET /live-footage} \mbox{} \\
Render a page that includes only the video stream from the camera.

\section{Camera endpoints}

\paragraph{POST /camera/address/set} \mbox{} \\
Endpoint that sets the camera network settings.

\verb|form:|
\begin{itemize}
    \item \verb|camera-ip|: the address of the connected camera
    \item \verb|camera-port|: the camera port
    \item \verb|camera-http-port|: the HTTP port of the camera
\end{itemize}

\verb|return:|
\begin{itemize}
    \item{
        HTTP code corresponds to the camera response:
        \begin{itemize}
            \item \verb|message|: description of the received response code from the camera
        \end{itemize}
    }
    \item{
        Possible HTTP 400: invalid supplied values
        \begin{itemize}
            \item \verb|message|: ``Invalid address!''
        \end{itemize}
    }
\end{itemize}

\paragraph{POST /camera/reboot} \mbox{} \\

Endpoint that reboots the camera.
\verb|return:|
\begin{itemize}
    \item{HTTP 200: success, empty message}
    \item{
    Other HTTP codes: HTTP code corresponds to the camera response
    \begin{itemize}
        \item \verb|"message"|: description of the received response code from the camera
    \end{itemize}
    }
\end{itemize}

