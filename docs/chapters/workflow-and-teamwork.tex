\section{Values and teamwork approach}
Our team has very specific values which we always aim to follow no matter the circumstances. This is how we ensure mutual trust and good understanding in the team.
The most important values we appreciate are hard work and authority. It goes without saying that we value hard work since we all aim to complete our project in the best possible way.
Authority is important to us because we should always respect our client and the university staff since they are the people who evaluate our work. Equally significant is respecting our chairman. After all we have chosen them to guide our meetings and should rely on them to do that.
Another crucial value in our team is humor. It is always better when working becomes a pleasure and not a boresome duty.

For our project, we try to direct a camera to a speaker in a room, using video and audio information. As video and audio information are considered to be personal and can be used with malicious intent, we are cautious and thoughtful with regard to the ethics of the system. The ethical implications of the system should be carefully reviewed to not allow any unethical abuse of the system. Below we explain our answer to each question in the Ethics Checklist to give more insight into our system and its potential use cases regarding personal data.

Does your planned activity involve processing of personal data?
Our project revolves around working with a microphone and a camera. Both of these process personal data in the forms of audio and video and transmit this so that it can be used by other people. Our system does not use any identifiable part of the data because we only utilize the location information of the speaker. Therefore, our system does not, in and of itself, process any personal data, but the hardware that we use together with the system does. For instance, the microphone, made in Germany by Sennheiser, processes audio from the room.
We can not modify the camera or microphone (or otherwise alter their behavior), so what we can do to mitigate the ethical risks of our project is to ensure we do not use the transmitted personal information.


Does it involve profiling, systematic monitoring of individuals, or processing of a large scale of special categories of data or intrusive methods of data processing?
Our tools (a camera and a microphone, conceptually) could technically be used in the surveillance of a room of people who have not consented to having video taken of them. We have access to the video camera stream, but we are not planning to store anything on the device after processing it locally. Our software alone therefore cannot be used to profile or monitor individuals. A person who has access to the microphone and the video stream is able to more accurately identify the speaking person and afterward store this information. Those ethical implications do not fall on us as the developers in this scenario.


Does this activity involve further processing of previously collected personal data?
We do not have access and are not allowed to use any of the data stored by our client’s company. The system could be extended with visual tracking that might require data from previously recorded people, but that data would be coming from publicly available, previously trained models. Apart from the visual tracking feature, there is no part of our system which requires previously stored data.

Is it planned to export personal data from the EU to non-EU countries? Specify the type of personal data and countries involved.
As explained above, the microphone and camera that we are using can be used to send personal data to any place in the world. This data involves video and audio of people. We can prevent the data obtained from going abroad by limiting the traffic from the network.

Is it planned to import personal data from non-EU countries into the EU or from a non-EU country to another non-EU country?
We do not plan on importing personal data from anywhere. We do not personally need the data obtained using the microphone and camera, so we will surely not import any data from non-EU countries.
Users of the cameras and microphones can use those devices to create personal data and send it abroad, but they could use any camera to do this and it does not pose any ethical risks besides the ones explained in the second question.


\section{Workflow practices}
\subsection{Approaches}
To ensure good workflow we decided to use Agile and more specifically - Scrum. Scrum gives teams the ability to cope with dynamic priorities and requirements by introducing sprint planning,retrospective and reviews. This way of planning helps organizing collaborative work faster and more precisely. Another advantage of this methodology is the ability to detect any issues right from the beginning and mitigate the risks related to the workflow.
We also decided to employ TDD(test-driven development) which integrates well with Scrum and has been shown to improve test coverage. It is also useful to write simple but yet reliable code.
Another practice we used was planning poker. We needed that to assign weights to issues we had come up with. However, this didn't prove to work well since our overall vision of the issues seemed to not be very accurate in the beginning.

\subsection{Requirements for new functionality}
In order to accept a new functionality we came up with three requirements:
\begin{enumerate}
    \item Has to have a test coverage of at least 80%
    \item Has to have a clear description of the functionality
    \item Mustn't break any other functionality
\end{enumerate}

\subsection{Definition of done}
Our project is considered done when all the listed must-have requirements are fulfilled. Any other requirements we manage to solve are a bonus. Realistically we will not be able to implement all of them but we aim to complete as many as possible.
We also have definition of done for all of our Gitlab issues. It describes the behavior that should be possible upon completion of this issue. Issues should be closed only after this definition of done has been satisfied.
\subsection{Additional tools}
\begin{itemize}
    \item Pylint is used for styling purposes
    \item Pytest for basic testing, pytest-cov to calculate coverage
    \item Mutmut for mutation testing
    \item git for version control using Gitlab as the server
    \item CI(continuous integration) for quality assurance - it will be ran on Gitlab after each change to verify correctness
\end{itemize}

\subsection{Communication}
We make sure all work is done on time and in an upstanding way by meeting every day. This happens either on campus, in Avonic's office or on Discord as a backup. We discuss our progress and help each other. If any problem arises we brainstorm and come up with solutions together.
Mattermost serves as a communication channel with our TA.
WhatsApp is our backup if everything else fails.

\section{Collaboration with Avonic}
