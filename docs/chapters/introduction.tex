With each oncoming year, online conferences and lectures are becoming more and more prominent\cite{computers-in-human-behavior}.
Having good equipment for them is a necessity for companies small and large,
universities and non-profit organizations alike.
This includes cameras, microphones, and software to combine the two into a good-looking and sounding stream.
For conferences and lectures of a larger scale, a static video camera in the back of the room may not suffice as it may be
hard for the audience to focus on what is happening, and individual speakers may not be recognizable.
Solutions that rely on movement\cite{chapel2020moving} or face tracking\cite{huang2018real-time-face-detection} exist,
but they become inaccurate when there are more people to capture (eg. two lecturers on opposite sides of the stage).
This is why we have been asked by Avonic to design and engineer a solution.


The objective of this report is to present and explain the details of our development process,
research and results of devising a system that directs a camera at the orator in a conference
using the Sennheiser TeamConnect Ceiling 2\cite{team-connect-2} microphone array and Avonic’s pan-tilt-zoom (PTZ) camera\cite{avonic}.
The most important finding from this report should be whether this implementation is feasible in a real commercial environment
and if it is more accurate and useful than using object tracking from the video stream directly.
We performed this experiment on behalf of Avonic in the span of ten weeks.


The report has the following structure:
\hyperref[ch:problem-analysis]{Chapter 2} will provide a problem analysis covering the technical and ethical aspects of the system.
The requirements and how they were changed with the development of the project are described in \hyperref[ch:requirements]{Chapter 3}.
\hyperref[ch:solution]{Chapter 4} will include the solution engineering process -- detail the solution we selected, its alternatives, and reasons for picking the one we did.
\hyperref[ch:workflow-and-teamwork]{Chapter 5} gives an insight into our team workflow and how we collaborated with Avonic.
A retrospective of the project is found in \hyperref[ch:reflection]{Chapter 6}.
Finally, the conclusion and key takeaways are stated in \hyperref[ch:conclusion]{Chapter 7}.
