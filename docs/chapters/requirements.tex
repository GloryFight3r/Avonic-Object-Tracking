This chapter presents an overview of the required functionality of the speaker tracking camera.
These requirements dictate the design of our system.
We have ordered them using the MoSCoW method\cite{moscow}, which we describe in \hyperref[ch:moscow-methodology]{Appendix D - MoSCoW methodology}.
All of the requirements fall under one of the following categories: ‘Must-have’,
‘Should-have’, ‘Could-have’, ‘Won't-have’. Furthermore,
we also list some non-functional requirements we have for our system.
Final requirements are listed in \hyperref[section:final-requirements]{Section 3.1}.
Finally, we also go through the evolution of the requirements in \hyperref[section:evolution-of-requirements]{Section 3.2}.
\hyperref[ch:requirements-appendix]{Appendix B - Requirements table} shows for each requirement
whether it has been completed or not.

\section{Final requirements}\label{section:final-requirements}
This section lists the requirements for the development of MaAT.

\subsection{Must have}
Role: The \textbf{user} must
\begin{itemize}
  \item Be able to send requests and receive acknowledgments from the Sennheiser TeamConnect Ceiling 2 microphone array.
  \item Be able to get the direction to the speaker’s location from the \gls{microphone} API.
  \item Be able to send requests to the Avonic CM-70 camera.
  \item Be able to point the camera in a certain direction.
  \item Be able to point the camera to specific pre-set locations, choosing the closest one to where the \gls{microphone} is
    currently pointing.
\end{itemize}
Role: The \textbf{system} must
\begin{itemize}
  \item Be able to follow the speakers, knowing the camera's and \gls{microphone}'s defined position, assuming that all speakers are
    located on the same plane.
  \item Have a web interface for easy use of the camera’s API.
\end{itemize}
\subsection{Should have}
Role: The \textbf{user} should
\begin{itemize}
  \item Be able to calibrate the system with data such as specifying the height of the microphone above the \gls{speaker-plane}.
\end{itemize}
Role: The \textbf{system} should
\begin{itemize}
  \item Be able to find the relative position between the camera and the \gls{microphone} provided only
    the height of the microphone and the \gls{speaker-plane}.
  \item Translate the calculated coordinates relative to the \gls{microphone} to coordinates the camera can point to.
\end{itemize}
\subsection{Could have}
Role: The \textbf{system} could
\begin{itemize}
  \item Be able to adapt the camera’s speed depending on how fast the speaker is moving.
  \item Make use of visual tracking to improve the positioning of the speaker in the frame.
  \item Enable the camera to zoom in appropriately depending on the speaker's distance from it.
\end{itemize}
\subsection{Won’t have}
Role: The \textbf{system} won't
\begin{itemize}
  \item Focus on two people if they talk at the same time.
  \item Work with multiple microphones.
\end{itemize}

\subsection{Non-functional requirements}
\begin{itemize}
  \item The project should be written in Python 3.10.
  \item The system should work with the Avonic CM-70 camera.
  \item The designed system should work on single-board computers.
  \item There should be more than 80\% test coverage.
  \item The system should be easy to install and \gls{calibration} can be
    done by teachers or presenters themselves, it should not require any technical knowledge or external devices.
\end{itemize}
\section{Evolution of requirements}\label{section:evolution-of-requirements}
Since the beginning of the project we have come up with few new requirements and have removed others we have deemed unnecessary.
A short list of the changes we have made is:
\begin{itemize}
  \item Added web user interface as a must-have requirement.
  \item Added visual tracking as a could-have requirement.
  \item Removed facial recognition as a could-have requirement.
\end{itemize}
