This chapter presents an overview of the required functionality of the speaker tracking camera. 
These requirements dictate the design of our system. 
We have ordered them using the MoSCoW method which we will describe in (4.1). 
All of the requirements fall under one of the following categories ‘Must-have’, 
‘Should-have’, ‘Could-have’, ‘Would-have’. We have listed all of the final requirements in (4.2).
Finally, we also go through the evolution of the requirements in (4.3).
\section{MoSCoW method}
Since we have a limited amount of time, we decided to prioritise the requirements in order to make progress 
and keep to deadlines. MoSCoW is a prioritisation technique for helping to understand and manage priorities. The letters stand
for 
\begin{itemize}
  \item \textbf{M}ust have - Needs to be implemented in order to have a working system. Even if a single Must Have requirements is not
    implemented the system is a failure.
  \item \textbf{S}hould have - Describe requirements that if missing would significantly affect the usability of the system, but
    for which a workaround exists.
  \item \textbf{C}ould have - Describe requirements that if missing do not have a dramatic impact on the usability of the system. They
    are nice to haves.
  \item \textbf{W}ont have - Describe requirements that are outside of the current scope of the project.
\end{itemize}
Furthermore, the requirements are first divided into 
\begin{itemize}
  \item Functional requirements - Describe what a product must do.
  \item Non-functional requirements - Describe how the system works.
\end{itemize}
\section{Final requirements}
\subsection{Must have}
Role: The \textbf{user} must
\begin{itemize}
  \item Be able to send requests and receive acknowledgments from the Sennheiser TeamConnect Ceiling 2 microphone.
  \item Be able to get the direction to the speaker’s location from the microphone API.
  \item Be able to send requests to the Avonic CM-70 camera.
  \item Be able to point the camera in a certain direction.
  \item Be able to point the camera to specific pre-set locations, choosing the closest one to where the microphone is 
    currently pointing.
\end{itemize}
Role: The \textbf{system} must
\begin{itemize}
  \item Be able to follow the speakers, knowing the camera's and microphone's defined position, assuming that all speakers are
    located in the same plane.
  \item Have a web interface for easy use of the camera’s API.
\end{itemize}
\subsection{Should have}
Role: The \textbf{system} should
\begin{itemize}
  \item Be able to find the relative position between the camera and the microphone provided only 
    the height of the microphone and the speaker plane.
  \item Translate the calculated coordinates relative to the microphone to coordinates the camera can point to.
\end{itemize} 
Role: The \textbf{user} should
\begin{itemize}
  \item Be able to calibrate the system with data such as specifying the height of the microphone above the speaker plane.
\end{itemize}
\subsection{Could have}
Role: The system could
\begin{itemize}
  \item Be able to adapt the camera’s speed depending on how fast the speaker is moving.
  \item Add visual tracking6 to improve the positioning of the camera.
  \item Enable the camera to zoom in appropriately depending on the speaker’s distance from it.
\end{itemize}
\subsection{Wont have}
Role: The system won't
\section{Evolution of requirements}
