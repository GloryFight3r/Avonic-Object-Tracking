For our project, we are going to make use of video and audio; both can be
considered personal data. Since personal data can be used with malicious
intent, we are cautious and thoughtful with regard to the ethics of the system.
The ethical implications of the system should be carefully reviewed to disallow
any unethical abuse of the system. Below we cover five important topics related
to ethical use of personal data to give more insight into our system and its
potential use cases.
\paragraph{Processing of personal data} \mbox{} \\
Our project revolves around working with a microphone and a camera.
Both of these process personal data in the forms of audio and video and
transmit this so that it can be used by people. Our system does not use any
personal part of the data because we only utilize the location information from
the speaker. Therefore, our system does not, in and of itself, process any
personal data, but the hardware that we use together with the system does. For
instance, the microphone, made in Germany by Sennheiser, processes audio from
the room.

We can not modify the camera or microphone (or otherwise alter their behavior),
but what we can do to mitigate the ethical risks of our project is to ensure we
do not use the transmitted personal information ourselves
\paragraph{Profiling or systematic monitoring} \mbox{} \\
Our tools (a camera and a microphone, conceptually) could technically be used
in the surveillance of a room of people who have not consented to have video
taken of them. We have access to the video camera stream, but we are not
planning to store anything on the device after processing it locally.
Our software alone therefore cannot be used to profile or monitor individuals.
A person who has access to the microphone and the video stream is able to more
accurately identify the speaking person and afterward store this information.
Those ethical implications do not fall on us as the developers in this
scenario.
\paragraph{Processing of previously collected data} \mbox{} \\
We do not have access and are not allowed to use any of the data stored by
Avonic. The system has been extended with visual tracking that required data
from previously recorded people, but that data was sourced from publicly
available, previously trained models. Apart from the visual tracking feature,
there is no part of our system which requires previously stored data.
\paragraph{Exporting to non-EU countries} \mbox{} \\
As explained above, the microphone and camera that we are using can be used to
send personal data to any place in the world. This data involves video and
audio of people. We can prevent the data obtained from going abroad by limiting
the traffic from the network.
\paragraph{Importing of personal data} \mbox{} \\
We do not plan on importing personal data from anywhere. We do not personally
need the data obtained using the microphone and camera, so we will surely not
import any data from non-EU countries. Users of the cameras and microphones can
use those devices to create personal data and send it abroad, but they could
use any camera to do this, and it does not pose any ethical risks besides the
ones explained in \hyperref[]{section 2.4.2}. % TODO
