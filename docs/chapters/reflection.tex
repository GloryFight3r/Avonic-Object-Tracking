In this chapter we reflect on both the process and the result of our project.
When we discuss the process, we focus on things that could have been done
better by our team in terms of practices.
When we discuss technological improvements, we focus on what can be changed in
order to improve the overall performance of our tracker.

\section{Process reflection}
Many things went well in the process of creating the speaker tracker. However,
there are two points of improvement that we want to reflect on. We also
highlight our testing practice because of how helpful it was during the
entirety of the project.

\paragraph{Production server} \mbox{} \\
As discussed in \hyperref[subsection:breaking-production-server]{section 5.5.2}
we had some problems with the production server. This was because we were
sometimes using the development server to test our code without realizing that
it works fundamentally differently from the production server. The production
server uses multi-processing, while the development server does not. This caused
problems with sharing data between threads.
After encountering this issue, we started using the production server for
everything. If we had done so from the start, we could have avoided many issues
related to threading.

\paragraph{Access to SBC} \mbox{} \\
It took eight weeks untill we got access to the NVIDIA Jetson. This meant that we
had a bit more than one week left to fix the issues that arose from deploying
our software on the SBC for the first time. For access to the NVIDIA Jetson, we
had to wait on Avonic to provide us with one. Therefore, the problem was mostly
out of our control.
If we had had access to the SBC at an earlier stage of the project, any bugs
related to the SBC would have been caught earlier in the project.

\paragraph{Testing} \mbox{} \\
Throughout the project, we always tested our code before merging into the main
branch on GitLab. We were dedicated to keeping the test coverage above 80\%.
This high coverage helped us to make sure no other functionality was broken when
new functionality was added. Many bugs have been caught this way before they
were werged.

\section{Technological improvements}
If Avonic wants to continue to work on our tracker, we would advice them to
focus on two improvements. Both of them are related to the quality of the video
stream. In this section we discuss how these changes would help to improve
our speaker tracker.

\paragraph{Tracking with adaptive zoom} \mbox{} \\
The audio model described in \hyperref[subsection:tracking]{section 4.5.3} uses
adaptive zooming. However, the constant zooming is not pleasant to look at.
Sometimes the speaker goes out of view, because the zooming increases the impact
of inaccurate microphone directions. Even when the speakers stays in view, the
zooming may be distracting to the viewers, because it can happen too fast and
too often.
For future improvements of the audio model, we would suggest to make the zoom
speed depend on the change of the zoom. Another way to improve the adaptive
zooming would be to change the zoom less often. This would make the video stream
more pleasant to watch.

\paragraph{Camera footage} \mbox{} \\
In our project we use OpenCV to read the camera footage from the camera's RTSP
stream. This way of reading the stream causes a delay. This delay also impacts
the object tracking models.
As Mr. Kahawati told us, Avonic has developed a much faster way of reading the
camera stream so the delay is negligible. Replacing OpenCV with this faster
method would make object tracking more responsive.
